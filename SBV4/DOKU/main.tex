\documentclass[12pt,german]{article}
\usepackage{listings}
%\usepackage[utf8]{inputenc}
\usepackage{inputenc}
\usepackage{graphicx}
\usepackage{float}
\usepackage{array}
\usepackage{pdfpages}

\lstset{
extendedchars=\true,
language=JAVA,
numbers=left, 
numberstyle=\footnotesize, 
%inputencoding=utf8,
%basicstyle=\ttfamily,
%basicstyle=\ttfamily\fontsize{8}{8},
%commentstyle=\ttfamily\fontsize{8}{8},
basicstyle=\tiny;
columns=fullflexible,
%xleftmargin=5pt,
frame=single,
breaklines=true,
postbreak=\mbox{{$\hookrightarrow$}\space},
}
\renewcommand{\thesubsubsection}{\alph{subsubsection} )}
%\renewcommand{\thesubsubsection}{\thesubsection.\alph{subsubsection} )}

\setcounter{section}{3}

\begin{document}

\title{Übungsaufgaben IV, SBV1 }
\author{Lukas Fiel, Lisa Panholzer}
\maketitle


\newpage
\section{Übungsaufgaben IV}
\subsection{Region Growing}

\subsubsection{Manuelles Image Growing}
todo

\subsubsection{Image Growing mit Labeling}
todo


\subsection{Optimaler Threshold}

\subsubsection{Adaptiver optimar Threshold}
todo

\subsubsection{Image Growing mit Labeling}
todo


\subsection{Objekterkennung mittels mathematischer Morphologie}

\subsubsection{Rechtecks Erkennung - binär}
mittels Erosion Dilation
bzww
arithmetische Operationen
todo

\subsubsection{Image Growing mit Labeling - Graustufen}
todo

\end{document}
